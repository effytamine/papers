\documentclass{article}

% Packages
\usepackage{graphicx} %  was not used here
\usepackage{amsmath} % american math society symbols, notations
\usepackage{amsfonts} % addon to ams
\usepackage{mathtools} % more notations
\usepackage[margin=1in]{geometry} % page marhin
\usepackage{fancyhdr} % layout and design
\usepackage{enumerate} % was not used
\usepackage[shortlabels]{enumitem} % was not used
\usepackage[makeroom]{cancel} % cleaner strikethrough

% typeset preferences
\newcommand{\paren}[1]{\mathopen{}\left( {#1}_{{}_{}}\,\negthickspace\right)\mathclose{}} % auto-scale parentheses including nested
\newcommand{\bracket}[1]{\mathopen{}\left[ {#1}_{{}_{}}\,\negthickspace\right]\mathclose{}} % auto-scale brackets including nested

% formatting
\pagestyle{fancy}
\fancyhead[l]{Sean Russell Villeza}
\fancyhead[c]{Section 2-4, Modern College Algebra by Vance}
\fancyhead[r]{\today}
\fancyfoot[r]{\today}
\renewcommand{\headrulewidth}{0.2pt}
\setlength{\headheight}{15pt} %Sets enough space for the header
\allowdisplaybreaks

% begin 
\begin{document}

\section{Introduction}

This paper serves as a compilation for expositions on the problems presented in section
2-4 of the book Modern College Algebra by Vance. The answers are written loosely 
and note-like.

\section{Problems}

Let us recall the definition of the \emph{axiom 6A} , the existence of additive 
inverses. The axiom states that;

\[
    \forall a \in \mathbb{Z}, \exists !x \left( a + x = 0\right)
\]

and that x is a unique solution to that equation denoted as -a for any a in the
 same definition (that is, \(\left(a + (-a) = 0\right)\)). We also take note of the \emph{theorem 2-12} which states; 

\[
    \forall a, \forall c \paren{ ac =  0 \to (a= 0 )\lor(c = 0)}
\] 

With these axioms in mind, the statement that 0 is the additive inverse of itself is 
true. Not only is it's existence justified by the additive identity, but also 
logically, the implication is satisfied for \(0 + x = 0\) where the unique 
solution \(x = 0\) is determined to be true. A more rigorous proof comes from 
deriving an equation by the axiom that is the addition for equality. It turns 
\(a = -a\) to \(a +a = 0\) by addition for equality, then by the distributive 
axiom \(a(1 + 1)  =0\) to be \(2a = 0\). By theorem 2-12 we deduce \(a = 0\).

Section 2-4 introduces \emph{multiplicative inverses}, that is more commonly known as its
own separate operation called \emph{division}. This produces new interactions and parallels
with the additive inverse and related axioms that take the form of a separate operation called 
called \emph{subtraction}. Particularly in identifying solutions that come out of shifting and
scaling in the real number line. The anatomy of a multiplication expression is defined as;
\[
    \forall a, b, c \in \mathbb{Z} \left(ab = c \right)
\]
\begin{center}
    where a and b are determined to be factors of c, and c is a multiple of a and b.
\end{center}

More complex factors are now available to be expressed such as the product of sums. Their solutions vary
depending on the internal structure of the sum. For instance, let us take \( (a+1)(a-1) = 0\) that is
the product of a + 1 and a - 1, particularly simplify its form. Using the
distributive axiom of multiplication over addition, the commutative and associative
axioms for addition and multiplication; 
\begin{align*}
    a\paren{a-1} + 1\paren{a-1} \\
    \paren{\paren{a}\paren{a} - \paren{a}\paren{1}} + \paren{\paren{1}\paren{a} - \paren{1}\paren{1}} \\
    (a)(a) + (-a)(1) + (1)(a) + (-1)(1) \\ 
    (a)(a) + \paren{(-a)(1) + (1)(a)} + (-1)(1) \\ 
    (a)(a) + \paren{(1)(a + (-a))} + (-1)(1) \\ 
    (a)(a) + \paren{(1)(0)} + (-1)(1) \\ 
    (a)(a) + (0) + (-1)(1) \\
    (a)(a) + (-1)(1) \\
    (a)(a) -1 \\ 
    (a)(a) \coloneqq a^2 \\
     a^2 - 1  
\end{align*}
This provides the notation we use for factors that are equal to each other, that is 
the \emph{power} notation, \(\forall a \in \mathbb{R}\) that \(a \cdot x\) where 
\(a = x\) is \(a^2\) or more generally;
\[
    a^n
\]
\begin{center}
    where n is the amount of factors equal to itself.
\end{center}

The product of sums in the form \((a^2 - b^2)\) from above is a special type of product 
known as the \emph{difference of squares}. By turning the expression into an equation equal to
0, we define a task to determine its solutions or \emph{roots}. In the context of factors, we manipulate 
the equation to give us what makes each factor collapse to 0, or literally equal to 0;
\begin{align*}
    \paren{a^2 - 1} &= 0 \\ 
    \paren{a+1} \paren{a-1} &= 0 
\end{align*}
\[
\begin{array}{rl@{\qquad}rl}
    \paren{a+1} &= 0 & \paren{a-1} &= 0 \\
    a + 1 &= 0 & a - 1 &= 0 \\ 
    a &= -1 & a &= 1 
\end{array}
\]
\begin{center}
    The solution to the difference of squares that is \((a^2-1) \text{ is } a =\{1, -1\}\). 
\end{center}

Finding solutions to expressions involving product of sums can be done using an array of methods. The method
used previously is called \emph{factoring}. This introduces a plethora of inquiry especially in this section
concerned with introduction to division. For example the proposition \(\exists a \in \mathbb{R} \paren{(ax = 1) \land (a = x)}\)
, that states it is true there is a number \(a\) that is, it is the multiplicative inverse of itself. Using algebraic 
manipulations we find that using that;
\begin{align*}
    ax &= 1 \\
    a &= x \\
    a \cdot a &\coloneqq a^2 \\
    a^2 - 1 &= 0 \\ 
    \text{by factorization; } a &= \{1, -1\} 
\end{align*}
this is a fundamental result in algebra. Just like how addition and subtraction has a special relationship
with 0. 0 is a number that is its own additive inverse -- a fact guaranteed by the additive identity axiom. 
On the other hand, we see a special relationship in multiplication and division, that is,
there exists a number that is its own multiplicative inverse. However unlike the two former operations,
multiplication admits to \emph{two} such integers, 1 and -1.  

It is clear there are parellels existing between addition and subtraction, and division and multiplication. 
In addition to notable differences previously. This prompts us to look at the behavior of each 
axiom towards division. Addition and multiplication possess the commutative and associative axiom. Subtraction 
and division cannot possess these axioms since these operations themselves are born out of different axioms. 
Commutativity axiom states that;
\begin{align*}
    \forall a, b \in \mathbb{R} (a\cdot b = b \cdot a) \text{ for multiplication; } \\
    \forall a, b \in \mathbb{R} (a + b = b + a) \text{ for addition; } 
\end{align*} 
While the associative axiom states that;
\begin{align*}
    \forall a, b, c \in \mathbb{R} \paren{(a\cdot b) \cdot c = a \cdot (b \cdot c)} \text{ for multiplication; } \\
    \forall a, b, c \in \mathbb{R} \paren{(a + b) + c = a + (b + c)} \text{ for addition; } 
\end{align*} 
The restrictions come particularly, in the order and grouping in which the inverses affect the final result. 
Division and subtraction themselves are by concept, just convenient notations for additive and multiplicative inverses. This is a concept
stressed about in fundamental college algebra. To show that division is not commutative is to prove that
\[
    \exists a,b \in \mathbb {Q}\paren{\frac{a}{b}\neq\frac{b}{a}}
\]
\begin{center}
    For \(\mathbb{Q}\) is the minimum viable number system that satisfies multiplicative inverses.
\end{center}
\begin{align*}
    \frac{a}{b} &= \frac{b}{a} \\
    \frac{a}{b} - \frac{b}{a} &= 0\\
    \frac{(a)(a) - (b)(b)}{ab} &= 0 \\
    \frac{a^2- b^2}{ab} &= 0 \\
    \cancel{(ab)}\frac{a^2- b^2}{\cancel{ab}} &= 0 (ab)\\
    a^2-b^2 &= 0
\end{align*}
\begin{center}
    by factorization; \((a+b)(a-b)\), that is \((a = b) \lor (b = -a)\). 
\end{center}

It shows the only two instances where commutativity in division is true. Only when a is equal to b
or when one is the additive inverse of each other, those that yield 0. It is sufficient to say that 
if there are only two distinct instances where commutativity in division is true, then all other cases
must render commutativity false, the negated proposition from above that has a similar truth value;
\[
    \forall a,b \in \mathbb{Q}\paren{\frac{a}{b} = \frac{b}{a}} 
\]
is proven false. Therefore division is not commutative. 

Similarly with the associativity axiom, in order to prove that division is not associative we have to
show that;
\[ 
    \exists a, b, c \in \mathbb{Q} \paren{\frac{\frac{a}{b}}{c} \neq \frac{a}{\frac{b}{c}}}
\]
\begin{align*}
    \paren{\frac{b}{c}}\paren{\frac{\frac{a}{b}}{c}} &= \paren{\frac{a}{\cancel{\frac{b}{c}}}} \paren{\cancel{\frac{b}{c}}} \\
    \paren{\cancel{c}}\paren{\frac{\paren{\frac{b}{c}}\paren{\frac{a}{b}}}{\cancel{c}}} &= (a)(c) \\
    \paren{\frac{\cancel{b}}{c}} \paren{\frac{a}{\cancel{b}}} &= ac \\
    \paren{\cancel{c}} \paren{\frac{a}{\cancel{c}}} &= \paren{ac} \paren{c} \\
    \paren{\frac{1}{\cancel{a}}} \paren{\cancel{a}} &= c^2 \\
    0 &= c^2 - 1
\end{align*}
\begin{center}
    by factorization; \((c+1)(c-1)\), that is \((c = 1) \lor (c = -1)\). 
\end{center}

Similar to proving that division is not commutative, it states that division only satisfies the 
associative axiom whenever c = 1 or -1. Hence for all other \(a, b, c \in \mathbb{Q}\) where c \(\neq\) 1 or -1, the 
proposition to show earlier, is true.
\end{document}
