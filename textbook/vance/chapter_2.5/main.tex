\documentclass{article}

\usepackage{amsmath} % american math society symbols, notations
\usepackage{amsfonts} % addon to ams
\usepackage{amssymb}
\usepackage{mathtools} % coloneqq
\usepackage[margin=1in]{geometry} % page margin
\usepackage{fancyhdr} % layout and design
\usepackage{enumerate} %These two package give custom labels to a list
\usepackage[shortlabels]{enumitem} 
\usepackage[makeroom]{cancel} % cleaner strikethrough

% typeset preferences
\newcommand{\paren}[1]{\mathopen{}\left( {#1}_{{}_{}}\,\negthickspace\right)\mathclose{}} % auto-scale parentheses including nested
\newcommand{\bracket}[1]{\mathopen{}\left[ {#1}_{{}_{}}\,\negthickspace\right]\mathclose{}} % auto-scale brackets including nested
\newcommand{\braces}[1]{\mathopen{}\left\{ {#1}_{{}_{}}\,\negthickspace\right\}\mathclose{}}

% formatting
\pagestyle{fancy}
\fancyhead[l]{Sean Russell Villeza}
\fancyhead[c]{Section 2-5, Modern College Algebra by Vance}
\fancyhead[r]{\today}
\fancyfoot[r]{\today}
\renewcommand{\headrulewidth}{0.2pt}
\setlength{\headheight}{15pt} % Sets enough space for the header
\allowdisplaybreaks

\begin{document}
\section{Introduction}  
This paper presents a collection of answers with their respective expositions to the problems 
presented in section 2-5 from the book Modern College Algebra by Vance. Section 2-5 introduces 
the definition and coming of rational and irrational numbers. Unified together, the set 
\(\mathbb{Q}\) and set \(\mathbb{Q}'\) makes up the real number system \(\mathbb{R}\). 

When a set satisfies all of the previous axioms given from the previous chapters then the set is already 
considered as a \emph{field}. By this logic the set of rational numbers \(\mathbb{Q}\) is already 
considered a field. However our work in Algebra is far too big and detailed as such the set of 
rational numbers are merely not enough. The introduction of irrational numbers once again called 
for inquiries regarding its interaction with the field axioms. 

\section{Problems}
\begin{enumerate}[start = 8, label = {\bfseries Problem \arabic*:}, leftmargin=1in] 
    \item Prove that the set of rational numbers \(\mathbb{Q}\) is closed under multiplication, that is, if x
        = \(\frac{n}{m}\) and \(y = \frac{p}{q}\) for integers \(n, m, p, \text{and } q\) then 
        \(x\cdot y\) can be expressed as the quotient of two integers. 
    
        \vspace{1em}
        Consider the following \(n, m, p, q \in \mathbb{Z}\), so that \(x = \frac{n}{m}\ \in 
        \mathbb{Q}\) and \(y = \frac{p}{q} \in \mathbb{Q}\)

        \(\implies x \cdot y = \frac{n \cdot p}{m \cdot q}\) by associativity and commutativity axioms
        
        if \(n \in \mathbb{Z}\) and \(p \in \mathbb{Z} \implies n \cdot p \in \mathbb{Z}\)
        by Axiom 1M that ensures multiplication between two integers are closed in the set
        of integers \(\mathbb{Z}\)

        similar to that \( \implies m \cdot q \in \mathbb{Z}\) should also be true for \(m, q \in Z\)

        rational numbers are defined as;
        \[
            \mathbb{Q} = \braces{\frac{n}{m} \mathrel{\Big|} \text{ for } n \text{ and } m \in \mathbb{Z} \text{ and } m \neq 0}
        \]

        if \(\paren{n \cdot p \in \mathbb{Z}} \land \paren{m \cdot q \in \mathbb{Z}} \implies 
        \frac{n \cdot p}{m \cdot q} \in \mathbb{Q}\) by definition of rational numbers 

        if \(x \cdot y = \frac{n\cdot p}{m \cdot q} \implies x \cdot y \in \mathbb{Q}\) for \(m, q \neq 0\)
        \[
        \therefore x\cdot y \in \mathbb{Q}
        \]

    \item Show that the additive inverse of a rational number is a rational number, that is, if \(x = 
        \frac{n}{m} \text{ for integers } n, m\), then \(-x\) can be expressed as the quotient of two integers
        
        \vspace{1em}
        Suppose the following \(x = \frac{n}{m} \text{ for } n, m \in \mathbb{Z}\)
        
        the definition of additive inverse states that;
        \[
            \forall x \in \mathbb{Z}, \exists y \in \mathbb{Z}(y \coloneqq -x \implies x + (-x) = 0)
        \]
        since it is true that \(n \in \mathbb{Z} \text{ then } -n \in \mathbb{Z}\) must also be true as the unique
        solution to the equation \(n + -n\) = 0 
        \begin{align*}
            \implies -(x) &= -(\frac{n}{m}) \\
            -x &= -\frac{n}{m} \text{ by multiplication for equality axiom}
        \end{align*}
        \(-\frac{n}{m}\) can be expressed as \(\frac{-n}{m}\) by the commutativity and associativity axioms

        for \(x, x + -x = 0 \implies \frac{n}{m} + (\frac{-n}{m}) = 0\) that is true for additive inverses
        
        using associativity and commutativity axioms we simplify the equation to \(\frac{n + (-n)}{m} = 0\)

        since \(n, -n \in \mathbb{Z} \implies n + (-n) \in \mathbb{Z} \text{ by the closure property of addition axiom}\)

        if \(\paren{n + (-n) \in \mathbb{Z}} \land \paren{m \in \mathbb{Z}} \implies \frac{n + (-n)}{m} \in 
        \mathbb{Q}\), so  
        \begin{align*}
            \frac{n}{m} + \frac{-n}{m} &= 0 \\
            \frac{n + (-n)}{m} &= 0 \\
            \frac{0}{m} &= 0 \\
            0 &= 0 \text{ is true}
        \end{align*}
        \(\implies \frac{-n}{m} = -x\) is the additive inverse for \(\frac{n}{m} = x\) 

        finally, since \(n, m, -n \in \mathbb{Z} \implies \paren{x = \frac{n}{m} \in \mathbb{Q}} \land 
        \paren{-x = \frac{-n}{m} \in \mathbb{Q}}\) from the definition of rational numbers. All conditions are 
        satisfied within \(\mathbb{Q}\) 
        \[
            \therefore \text{ for } x \in \mathbb{Q} \implies -x \in \mathbb{Q} 
        \]
    \item Show that the set \(\mathbb{Q}\) is closed under addition, that is, if \(x = \frac{n}{m}\) and 
    \(y = \frac{p}{q}\) for integers \(n, p\) and for integers \(m, q \neq 0\), then \(x + y\) can be 
    expressed as a quotient of an integer by a non zero integer.
    
    \vspace{1em}
    Suppose \(x = \frac{n}{m} \text{ and } y = \frac{p}{q}\) for \(\paren{n, p \in \mathbb{Z}} \land \paren{m, q 
    \in \mathbb{Z} \neq 0}\):

    \(\implies x + y = \frac{n}{m} + \frac{p}{q} \) must be true;
    \begin{align*}
            x + y &= \frac{n}{m} + \frac{p}{q} \\
            x + y &= \frac{n\cdot q + p\cdot m}{m\cdot q} \text{ by  commutativity and associativity axioms;}
    \end{align*}
    if \(n, q \in \mathbb{Z} \implies n \cdot q \in \mathbb{Z}\) by the closure axiom;
    
    and if \(p, m \in \mathbb{Z} \implies p \cdot m \in \mathbb{Z}\) by the closure axiom;

    and also that \(\implies m \cdot q \in \mathbb{Z}\);

    with all of that, \(\paren{n\cdot q + p\cdot m \in \mathbb{Z}} \land \paren{m\cdot q \in \mathbb{Z}} \implies 
    \frac{n\cdot q + p\cdot m}{m\cdot q} \in \mathbb{Q}\) by the definition of rational numbers;
    
    if \(\frac{n\cdot q + p\cdot m}{m\cdot q} \in \mathbb{Q} \implies x + y \in \mathbb{Q}\);
    
    \[
        \forall x \in \mathbb{Q}, \forall y \in \mathbb{Q} \paren{x + y \in \mathbb{Q}} \therefore 
        \mathbb{Q} \text{ is closed under addition.}
    \]

    \item Show that if x is in \(\mathbb{R} \setminus \mathbb{Q}\), then \(-x\) is also in \(\mathbb{R} \setminus \mathbb{Q}\), and also 
    \(\frac{1}{x}\) is in \(\mathbb{R} \setminus \mathbb{Q}\). 

    \vspace{1em}
    \textbf{Theorem 11-1:} If \(\paren{x \in \mathbb{R \setminus Q}} \implies \paren{-x \in \mathbb{R \setminus Q}}\) 

    Consider \(x \in \mathbb{R} \setminus \mathbb{Q}\) and assume\(-x \in \mathbb{Q}\);
    
    we now recall the definition of additive inverses; 
    \[
        \forall x \in \mathbb{Z} \implies \exists(-x) \in \mathbb{Z} \text{ where } x + -x = 0; 
    \]
    from that, theorem 2-1 states that for any \(-x\) that is the additive inverse of 
    \(x\), \(-(-x) = x\). \(x\) the additive inverse of the additive inverse of a number is itself.
    this is also the basis for closure of numbers under negation.

    It follows that by a set inclusion's property of transitivity \(\mathbb{Z} \subsetneq \mathbb{Q} \subsetneq \mathbb{R}\), the set of real
    numbers particularly under which are rational numbers, should effectively possess the axioms aforementioned;
    \[
        \forall x = \frac{a}{b} \in \mathbb{Q} \text{ for } a, b \in \mathbb{Z} \implies \exists y = -x = \frac{-a}{b} \in \mathbb{Q} 
    \]

    Since \(-x \in \mathbb{Q} \implies -(-x) = x \in \mathbb{Q}\);

    but \(x \in \mathbb{R} \setminus \mathbb{Q}\) is already true, so it contradicts the assumption\(-x \in \mathbb{Q}\)
    \[
        -x \in \mathbb{R} \setminus \mathbb{Q} \therefore \text { The additive inverse of an irrational number is also an 
        irrational number} 
    \]
    
    \textbf{Theorem 11-2:} If \(\braces{\paren{x \in \mathbb{R \setminus Q}} \land \paren{x \neq 0}} \implies \paren
    {\frac{1}{x} \in \mathbb{R \setminus Q}}\)  
    
    Suppose \(x \in \mathbb{R \setminus Q}\) and lets assume \(\frac{1}{x} \in \mathbb{Q}\);

    recalling the definition of a rational number;
    \[
        \mathbb{Q} = \braces{\frac{a}{b} \mathrel{\Big|} \text{ for } a, b, \in \mathbb{Z} \text{ and } b \neq 0}
    \]
    that is derived from the definition of the multiplicative inverses;
    \[
        \forall x \in \mathbb{Q}, \exists y\paren{y = \frac{1}{x} \in \mathbb{Q} \implies x\cdot \frac{1}{x} = 1} 
    \]
    Since rational numbers are made up of integers, and integers are closed under multiplication, it follows that 
    through set inclusion property of transitivity \(\mathbb{Z} \subsetneq \mathbb{Q}\) rational numbers are also 
    closed within multiplication.

    from all of this, the inverse of a multiplicative inverse must also be the number itself \(\frac{1}{\frac{1}{x}} = x\)
    
    so \(\frac{1}{\frac{1}{x}} \in \mathbb{Q} = x \in \mathbb{Q}\)

    however, this contradicts the fact that \(x \in \mathbb{R \setminus Q}\), so our assumption \(\frac{1}{x} \in \mathbb{Q}\) 
    must be false;
    
    so; \(\frac{1}{x} \in \mathbb{R \setminus Q} \text{ from } x \in \mathbb{R \setminus Q}\)
    \[
        \therefore \text{ The multiplicative inverse of an irrational number is also an irrational number}
    \]

    \item Show that if x is in \(\mathbb{Q}\) and y is in \(\mathbb{R \setminus Q}\) then \(x + y\) is in \(\mathbb{R 
    \setminus Q}\) and, if \(x \neq 0\) then \(x\cdot y\) is also in \(\mathbb{R \setminus Q}\). 

    \textbf{Theorem 12-1: } if \(\paren{x \in \mathbb{Q}} \land \paren{y \in \mathbb{R \setminus Q}} \implies x + y \in \mathbb{R \setminus Q}\)

    Let\(x \in \mathbb{Q} \text{ and } y \in \mathbb{R \setminus Q}\), assume \(x + y \in \mathbb{Q}\);

    it is true that \(\mathbb{Z}\) is closed under addition;
    
    it is also true that \(\forall x \in \mathbb{Q} \implies x = \frac{a}{b}\) where \(a \in \mathbb{Z} \setminus 0\);
    
    and by set inclusion transitivity property, it shows \(\mathbb{Z} \subsetneq \mathbb{Q}\);

    so it follows that axioms of \(\mathbb{Z}\) must also be in \(\mathbb{Q}\), particularly \(\mathbb{Q}\) must 
    also be closed under addition. 

    Using axioms applicable to \(\mathbb{Q}\), we find that;
    \begin{align*}
        x + y \in \mathbb{Q}, q \in \mathbb{Q} \\
        x + y &= q \\
        \paren{x + y} - x &= q - x \\
        \cancel{x} + y - \cancel{x} &= q - x \\
        y &= q - x \text{ we isolate y}.
    \end{align*}

    Since \(\mathbb{Q}\) is closed under addition, it must also be closed under subtraction as the inverse
    operation;
    
    and because \(q \in \mathbb{Q}\) and \(x \in \mathbb{Q}\) then \(q - x \in \mathbb{Q}\)

    so it follows that \(y = q - x \in \mathbb{Q}\) y \(\in \mathbb{Q}\) needs to be true.

    This contradicts \(y \in \mathbb{R \setminus Q}\), so our assumption \(x + y \in \mathbb{Q}\) must be false.

    \(x + y \in \mathbb{R} \text{ and } x + y \notin \mathbb{Q}, \text{ then } x + y \in \mathbb{R \setminus Q}\) 
    \[
        \therefore \text{ The sum of a rational number and an irrational number is also an irrational number}
    \]

    \textbf{Theorem 12-2:} if \(\paren{x \in \mathbb{Q} \setminus 0} \land \paren{y \in \mathbb{R \setminus Q}} \implies 
    xy \in \mathbb{R \setminus Q}\)

    Consider \(x \in \mathbb{Q}, x \neq 0 \text{ and } y\in \mathbb{R \setminus Q}\), we assume \(xy \in \mathbb{Q}\); 

    by definition of \(\mathbb{Q}\), rational numbers are \(\frac{a}{b} \text{ for } a \in \mathbb{Z}, b \in \mathbb{Z}\); 

    since integers are closed under multiplication, and \(\mathbb{Z} \subsetneq \mathbb{Q}\), it follows that 
    \(\mathbb{Q}\) is also closed under multiplication.

    Using axioms applicable to \(\mathbb{Q}\) we find that;
    \begin{align*}
        xy \in \mathbb{Q}, q \in \mathbb{Q} \\
        xy &= q \\
        \frac{\cancel{x}y}{\cancel{x}} &= \frac{q}{x} \text{ for } x \neq 0 \\ 
        y &= q \cdot \paren{\frac{1}{x}}
    \end{align*}

    Since \(q \in \mathbb{Q}\) and \(\frac{1}{x} \in \mathbb{Q}\) then \(q \cdot \frac{1}{x} \in \mathbb{Q}\)
    
    and because \(y = q \cdot \frac{1}{x} \in \mathbb{Q}\) then it needs to be true that \(y \in \mathbb{Q}\) 

    however, that contradicts the fact that \(y \in \mathbb{R \setminus Q}\), so our assumption must be false for \(xy 
    \in \mathbb{Q}\)

    \(x + y \in \mathbb{R} \text{ and } x + y \notin \mathbb{Q}\), then \(x + y \in \mathbb{R \setminus Q}\)
    \[
        \therefore \text{ The product of a rational number and an irrational number is also an irrational number}
    \]

    \item Outline a proof of the fact that there is no rational number who's cube is 2

    \textbf{Theorem 13-1:} \(\forall n, m\paren{(\frac{n}{m})^3 = 2 = x \implies x \in \mathbb{R \setminus Q}}\)

    Consider \(n, m\) such that \((\frac{n}{m})^3 = 2 = x \in \mathbb{Q}\)

    using field axioms applicable to \(\mathbb{Q}\);
    \begin{align*}
        \left(\frac{n}{m}\right)^3 &= 2 \\
        \frac{n^3}{m^3} &= 2 \\
        (\cancel{m^3}) \frac{n^3}{\cancel{m^3}} &=  2m^3 \\
        n^3 &= 2m^3 \text{ we isolate } n^3
    \end{align*}
    
    Since the set of integers are closed within multiplication, for \(\frac{n}{m} \in \mathbb{Q}\) then \(n \in 
    \mathbb{Z}\) must also be true;

    and because \(n^3 = 2m^3\), for \(2m^3\) is even and \(a\) multiple of 2, it follows that \(n\) must also be even
    as it is an integer;

    then \(\implies n^3 = \left(2k\right)^3\)

    Using field axioms again;
    \begin{align*}
        \left(2k\right)^3 &= 2m^3 \\
        8k^3 &= 2m^3 \\
        \frac{\cancel{8}k^3}{\cancel{2}} &= \frac{\cancel{2}m^3}{\cancel{2}} \\ 
        4k^3 &= m^3 \text{ we isolate } m^3
    \end{align*}

    Since \(m \in \mathbb{Z}\) due to \(\frac{n}{m}\in \mathbb{Q}\), then by closure property, \(m^3 = 4k^3\) means m
    must also be even for \(4k^3\) is a multiple of 2.

    \(\forall a, b \in \mathbb{Q} \frac{a}{b} | \text{ a and b are expressible in lowest terms}\), expressing in lowest 
    terms means for \(a, b \implies gcd(a,b) = 1\), that is, a and b must be relatively prime.

    because n is even, and m is even, then \(\frac{n}{m} \text{ in } \left(\frac{n}{m}\right)^3 = 2\) can never be reduced
    to its lowest terms, hence, never relatively prime;

    existence of such a number is impossible.

    It follows that our asumption \(n, m\) such that \(\left(\frac{n}{m}\right)^3 = 2 = x \in \mathbb{R \setminus Q}\) must 
    be false, that is;
    \[
        \left(\frac{n}{m}\right)^3 = 2 = x, \implies x \notin \mathbb{Q} \text{ must be true }
    \]
    Because \(x \in \mathbb{R}\), then \( x \in \mathbb{R \setminus Q}\) 
    \[
        \therefore \text{ There is no rational number whos cube is 2}
    \]

    \item Outline a proof of the fact that there is no rational number who's square is 3

    \textbf{Theorem 14-1: } \(\forall n, m\paren{\left(\frac{n}{m}\right)^3=2=x \implies x \in \mathbb{R \setminus Q}}\)

    Consider \(n, m\) such that \(\left(\frac{n}{m}\right)^2 = 3 = x \in \mathbb{Q}\)

    using field axioms applicable to \(\mathbb{Q}\)
    \begin{align*}
        \left(\frac{n}{m}\right)^2 &= 3 \\
        \frac{n^2}{m^2} &= 3 \\
        (\cancel{m^2}) \frac{n^2}{\cancel{m^2}} &= 3(m^2) \\
        n^2 &= 3m^2 \text{ we isolate } n^2 
    \end{align*}

    \(3m^2\) is odd, a multiple of 3. It is clear that since \(\left(\frac{n}{m}\right)^2 \in \mathbb{Q}\), and the set of
    odd integers is closed under multiplication, it follows that \(n \in \mathbb{Z} \text{ in } n^2\) must also be a 
    multiple of 3.

    Using field axioms again;
    \begin{align*}
        n^2 &= 3m^2 \\ 
        (3k^2) &= 3m^2 \\ 
        9k^2 &= 3m^2 \\ 
        \frac{\cancel{9}k^2}{\cancel{3}} &= \frac{\cancel{3}m^2}{\cancel{3}} \\
        3k^2 &= m^2 \text{ we isolate } m^2
    \end{align*}

    \(3k^2\) is odd, a multiple of 3 that is equal to \(m\). Because the set of integers are closed under multiplication, it 
    follows that \(m\) in \(m^2\) has to also be a multiple of 3.

     \(\forall a, b \in \mathbb{Q} \frac{a}{b} | \text{ a and b are expressible in lowest terms}\). Expressing in lowest terms 
     means that a and b must be relatively prime. Since \(\frac{n}{m}\) is never relatively prime for
     \(n\) is a multiple of 3 and \(m\) is also a multiple of 3, then it can never be expressed in lowest form, which does not 
     exist in for \(x \in \mathbb{Q}\).

     This contradicts our assumption \(x \in \mathbb{Q}\), so then \(x \notin \mathbb{Q}\) must be true.

     Since \(x \in \mathbb{R}\) and \(x \notin \mathbb{Q}\) then \(x \in \mathbb{R \setminus Q}\) 
    \[
        \therefore \text{ There is no rational number who's cube is 2 }
    \]

    \item Show that if a, b are elements of \(\mathbb{Q}\) and if a is not equal to 0, then there is a unique rational number 
    \(x\) which is a solution to the equation \(ax + b = c\)

    \textbf{Theorem 15-1:} \(\forall a, b, c \paren{a, b, c \in \mathbb{Q} \setminus 0 \implies ax + b = c}\) where x is unique;
    
    we solve for x using field axioms;
    \begin{align*}
        ax + b &= c \\
        ax &= c - b \\
        x &= \frac{c - b}{a} 
    \end{align*}

    if \(c \in \mathbb{Q}\) and \(b \in \mathbb{Q}\) then \(c - b \in \mathbb{Q}\) by closure property of \(\mathbb{Q}\)

    \(a \in \mathbb{Q}\) so then \(\frac{c - b}{a} \in \mathbb{Q}\), it follows that for \(x = \frac{c - b}{a} \implies x \in \mathbb{Q}\) must also be true.

    so \(x\) is a rational number.

    To show uniqueness, we assume \(x_1, x_2\) as solutions to \(ax + b = c\) where \(x_1 \neq x_2\)

    then \(ax_1 + b = c, ax_2 + b = c\) are both true;
    \begin{align*}
        \left(ax_1 + b\right) - \left(ax_2 + b\right) &= 0 \\
        ax_1 + \cancel{b} - ax_2 - \cancel{b} &= 0 \\ 
        ax_1 - ax_2 &= 0 \\
        a\left(x_1 - x_2\right) &= 0
    \end{align*}

    recall that \(a \neq 0\), then we conclude that \(x_1 - x_2 = 0\);

    but that contradicts \(x_1 \neq x_2\)
    
    we conclude that our assumption must be false, and that \(x_1 = x_2\) must be true.
    \[
        \therefore \text{ x is rational and the unique solution to } ax + b = c \in \mathbb{Q}
    \]

    \item Show that if p, q, and r are nonzero integers, then the real number \(\frac{p + q\sqrt{2}}{r}\) is irrational. 

    \textbf{Theorem 16-1:} \(\forall p, q, r\paren{p, q, r \in \mathbb{Z} \setminus 0 \implies \frac{p + q \sqrt{2}}{r} \in \mathbb{R \setminus Q}}\)
    
    Consider \(p, q, r \in \mathbb{Z}\setminus 0\) and assume \(\frac{p + q\sqrt{2}}{r} \in \mathbb{Q}\)

    Using field axioms applicable to \(\mathbb{Q}\)
    \begin{align*}
        (\cancel{r})\frac{p + q\sqrt{2}}{\cancel{r}} &= 0 (r) \\
        p + q\sqrt{2} &= 0 \in \mathbb{Q}, \text{ for } r \in \mathbb{Z} \setminus 0 \\
        q\sqrt{2} &= -p \in \mathbb{Q}, \text{ for } p \in \mathbb{Z} \\
        \sqrt{2} &= -\frac{p}{q}
    \end{align*}

    if \(-p \in \mathbb{Z}\), and it is true that \(q \in \mathbb{Z}\), then \(-\frac{p}{q} \in \mathbb{Q}\) but this 
    contradicts \(\sqrt{2} = - \frac{p}{q}\) for \(\sqrt{2} \notin \mathbb{Q}\) then the assumption \(\frac{p + q\sqrt{2}}{r}\) must be false.

    \(\frac{p+q\sqrt{2}}{r} \notin \mathbb{Q}\), but \(\frac{p+q\sqrt{2}}{r} \in \mathbb{R}\)
    \[
        \therefore \frac{p+q\sqrt{2}}{r} \text{ is irrational } 
    \]
\end{enumerate}
\end{document}
